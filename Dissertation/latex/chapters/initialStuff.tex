\chapter*{Proforma}

{\large
\begin{tabular}{ll}
% Name:               & \bf Henry Mattinson                      \\
Candidate Number:   & \bf 2393G                                \\
% College:            & \bf Christ's College                     \\
Project Title:      & \bf Music Generation in Microsoft Excel \\
Examination:        & \bf Computer Science Tripos -- Part II, June 2019  \\
Word Count:         & \bf 11444\footnotemark[1]  \\
Line Count:         & \bf 1510\footnotemark[2]  \\
Project Originator: & Prof. Alan Blackwell                    \\
Supervisor:         & Dr. Advait Sarkar                    \\
\end{tabular}
}
\footnotetext[1]{Computed by summing \texttt{texcount -1 -utf8 -sum -inc diss.tex} using flags to including words in tables over the five main chapters.}
\footnotetext[2]{File line count for all Typescript and Python files and the JSON file for customFunctions definitions. No typescript or node congif files, css, or markup included.}
\stepcounter{footnote}


\section*{Original Aims of the Project}

The main aim of the project was to create a system for music expression and playback allowing users to play individual notes and chords and define their durations, define multiple parts, play loops, define sequences of notes and chords and be able to call these for playback and define the tempo of playback. Followed by the implementation of a converter from an existing musical notation to the Excel system (with compression as an extension) and usability testing of the Excel system.

\section*{Work Completed}

I designed a notation for music expression in Excel and built a prototype (Excello) satisfying the success criteria above. Participatory design sessions with 21 users provided formative evaluation leading to the implementation of many additional features as extensions. I contributed part of my implementation to an open-source library, this has been merged and published. I built a converter from MIDI to the Excello notation which can convert exactly or perform compression. This was used to translate a corpus of music to the Excello notation. I performed summative evaluation with the users from the participatory design.

\section*{Special Difficulties}

None.

\newpage
\section*{Declaration}

\paragraph{} I, Henry Mattinson of Christ's College, being a candidate for Part II of the Computer Science Tripos,
hereby declare that this dissertation and the work described in it
are my own work, unaided except as may be specified below, and
that the dissertation does not contain material that has already
been used to any substantial extent for a comparable purpose.

\paragraph{} I, Henry Mattinson of Christ's College,
am content for my dissertation to be made available to the students and staff of the University.

\bigskip
\leftline{Signed [signature]}

\medskip
\leftline{Date [date]}

\tableofcontents

% \listoffigures

% \newpage
% \section*{Acknowledgements}
%
% \paragraph{} Advait Sarkar
%
% \paragraph{} Alan Blackwell
%
% \paragraph{} The suggestions, requests, bug-reportinga and intruige of my 21 participants were invaluable. It brought a lot of joy to see Excello being used by them for their own arrangements.\\
% \\
% Harri Bell-Thomas \\
% Thomas Edney \\
% Thomas Fisher \\
% Simon Fraser \\
% Ryan Gidda \\
% Oliver Hope \\
% Max Langtry \\
% Pao Maneepairoj \\
% Thomas Marge \\
% Franz Nowak \\
% Matteo Pozzi \\
% Luke Sheeran \\
% Dom Stafford \\
% Simeon Stoykov \\
% Thomas Strudwick \\
% Lucy Sun \\
% Rajan Troll \\
