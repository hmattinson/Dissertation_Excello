\newpage
\section{Repository Overview}

\tikzstyle{dir}=[draw=black,thick,anchor=west]
\tikzstyle{src}=[draw=black,thick,anchor=west,fill=red!30]
\tikzstyle{file}=[draw=black,thick,anchor=west,draw=none]

\begin{wrapfigure}{r}{8cm}
% \begin{figure}[tbh]
\begin{tikzpicture}[%
  grow via three points={one child at (0.5,-0.7) and
  two children at (0.5,-0.7) and (0.5,-1.4)},
  edge from parent path={(\tikzparentnode.south) |- (\tikzchildnode.west)}]
  \node [dir] {Excello}
    child { node [dir] {Excel Music}
      child { node [dir] {assets}}
      child { node [dir] {node\_modules}}
      child { node [src] {src}
        child { node [file] {index.ts}}
        child { node [file] {turtle.ts}}
        child { node [file] {chords.ts}}
        child { node [file] {regex.ts}}
        child { node [file] {conversions.ts}}
        child { node [file] {bracketsParse.ts}}
      }
      child [missing] {}
      child [missing] {}
      child [missing] {}
      child [missing] {}
      child [missing] {}
      child [missing] {}
      child { node [file] {index.html}}
      child { node [file] {manifest.xml}}
    }
    child [missing] {}
    child [missing] {}
    child [missing] {}
    child [missing] {}
    child [missing] {}
    child [missing] {}
    child [missing] {}
    child [missing] {}
    child [missing] {}
    child [missing] {}
    child [missing] {}
    child { node [dir] {CustomFunctions}
      child { node [src] {src}
        child { node [file] {customFunctions.ts}}
      }
      child [missing] {}
      child { node [dir] {node\_modules}}
      child { node [file] {manifest.xml}}
    }
    child [missing] {}
    child [missing] {}
    child [missing] {}
    child [missing] {}
    child { node [dir] {MIDI}
      child { node [file] {MIDI\_Conversion.ipynb}}
    };
\end{tikzpicture}
\caption{File structure overview showing original files}
\label{fig:repo}
\vspace{-30pt}
% \end{figure}
\end{wrapfigure}

\paragraph{} Figure \ref{fig:repo} shows a reduced project file structure including all orginal source code. The directory Excel Music contains all the code for the add-in that parses the notation and produces music. Both this and CustomFunctions were generator using the Yeomann generator. The manifest.xml files are added to Excel and point to the resources to run the add-in. node\_modelues contains all libraries required to run the add-ins and are managed using npm.

\paragraph{} The index.html file defines the window that appears on the right of the spreadsheet. assets contains the piano samples. index.ts defines what happens when the different buttons of the window are pressed and imports from the remaining Typescript files. turtle.ts contains all the code required to produce musical output from the spreadsheet of turtle definitions and notes in cells, with helper functions in regex.ts, conversions.ts and bracketsParse.ts. bracketsParse.ts was based on Parenthesis which was initially incompatible for Excello's needs. chords.ts is for inserting chord notes into the grid with the chord input tool.

\paragraph{} customFunctions.ts contains the implementation of the EXCELLO.TURTLE and EXCELLO.MODULATE functions. The index.html file created when generating this add-in is not seen by the user so was not re-written.

\paragraph{} MIDI\_Conversion.ipynb is a Python notebook for converting MIDI files to the Excello notation. This includes methods to automate the conversion of the MIDI corpora which are also included in the MIDI directory.

\paragraph{} I shall release Excel Music and CustomFunctions components as an open source project under the MIT license. This is compatable with the MIT licenses of Tone.js and tonal. The Salamander piano samples come under a creative commons license so credit shall be given in the Excel add-in window.
